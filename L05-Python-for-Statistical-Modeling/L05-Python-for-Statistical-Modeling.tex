
% Default to the notebook output style

    


% Inherit from the specified cell style.




    
\documentclass{article}

    
    \usepackage[adobefonts]{ctex}    
    \usepackage{graphicx} % Used to insert images
    \usepackage{adjustbox} % Used to constrain images to a maximum size 
    \usepackage{color} % Allow colors to be defined
    \usepackage{enumerate} % Needed for markdown enumerations to work
    \usepackage{geometry} % Used to adjust the document margins
    \usepackage{amsmath} % Equations
    \usepackage{amssymb} % Equations
    \usepackage{eurosym} % defines \euro
    \usepackage[mathletters]{ucs} % Extended unicode (utf-8) support
    \usepackage[utf8x]{inputenc} % Allow utf-8 characters in the tex document
    \usepackage{fancyvrb} % verbatim replacement that allows latex
    \usepackage{grffile} % extends the file name processing of package graphics 
                         % to support a larger range 
    % The hyperref package gives us a pdf with properly built
    % internal navigation ('pdf bookmarks' for the table of contents,
    % internal cross-reference links, web links for URLs, etc.)
    \usepackage{hyperref}
    \usepackage{longtable} % longtable support required by pandoc >1.10
    \usepackage{booktabs}  % table support for pandoc > 1.12.2
    \usepackage{ulem} % ulem is needed to support strikethroughs (\sout)
    

    
    
    \definecolor{orange}{cmyk}{0,0.4,0.8,0.2}
    \definecolor{darkorange}{rgb}{.71,0.21,0.01}
    \definecolor{darkgreen}{rgb}{.12,.54,.11}
    \definecolor{myteal}{rgb}{.26, .44, .56}
    \definecolor{gray}{gray}{0.45}
    \definecolor{lightgray}{gray}{.95}
    \definecolor{mediumgray}{gray}{.8}
    \definecolor{inputbackground}{rgb}{.95, .95, .85}
    \definecolor{outputbackground}{rgb}{.95, .95, .95}
    \definecolor{traceback}{rgb}{1, .95, .95}
    % ansi colors
    \definecolor{red}{rgb}{.6,0,0}
    \definecolor{green}{rgb}{0,.65,0}
    \definecolor{brown}{rgb}{0.6,0.6,0}
    \definecolor{blue}{rgb}{0,.145,.698}
    \definecolor{purple}{rgb}{.698,.145,.698}
    \definecolor{cyan}{rgb}{0,.698,.698}
    \definecolor{lightgray}{gray}{0.5}
    
    % bright ansi colors
    \definecolor{darkgray}{gray}{0.25}
    \definecolor{lightred}{rgb}{1.0,0.39,0.28}
    \definecolor{lightgreen}{rgb}{0.48,0.99,0.0}
    \definecolor{lightblue}{rgb}{0.53,0.81,0.92}
    \definecolor{lightpurple}{rgb}{0.87,0.63,0.87}
    \definecolor{lightcyan}{rgb}{0.5,1.0,0.83}
    
    % commands and environments needed by pandoc snippets
    % extracted from the output of `pandoc -s`
    \providecommand{\tightlist}{%
      \setlength{\itemsep}{0pt}\setlength{\parskip}{0pt}}
    \DefineVerbatimEnvironment{Highlighting}{Verbatim}{commandchars=\\\{\}}
    % Add ',fontsize=\small' for more characters per line
    \newenvironment{Shaded}{}{}
    \newcommand{\KeywordTok}[1]{\textcolor[rgb]{0.00,0.44,0.13}{\textbf{{#1}}}}
    \newcommand{\DataTypeTok}[1]{\textcolor[rgb]{0.56,0.13,0.00}{{#1}}}
    \newcommand{\DecValTok}[1]{\textcolor[rgb]{0.25,0.63,0.44}{{#1}}}
    \newcommand{\BaseNTok}[1]{\textcolor[rgb]{0.25,0.63,0.44}{{#1}}}
    \newcommand{\FloatTok}[1]{\textcolor[rgb]{0.25,0.63,0.44}{{#1}}}
    \newcommand{\CharTok}[1]{\textcolor[rgb]{0.25,0.44,0.63}{{#1}}}
    \newcommand{\StringTok}[1]{\textcolor[rgb]{0.25,0.44,0.63}{{#1}}}
    \newcommand{\CommentTok}[1]{\textcolor[rgb]{0.38,0.63,0.69}{\textit{{#1}}}}
    \newcommand{\OtherTok}[1]{\textcolor[rgb]{0.00,0.44,0.13}{{#1}}}
    \newcommand{\AlertTok}[1]{\textcolor[rgb]{1.00,0.00,0.00}{\textbf{{#1}}}}
    \newcommand{\FunctionTok}[1]{\textcolor[rgb]{0.02,0.16,0.49}{{#1}}}
    \newcommand{\RegionMarkerTok}[1]{{#1}}
    \newcommand{\ErrorTok}[1]{\textcolor[rgb]{1.00,0.00,0.00}{\textbf{{#1}}}}
    \newcommand{\NormalTok}[1]{{#1}}
    
    % Additional commands for more recent versions of Pandoc
    \newcommand{\ConstantTok}[1]{\textcolor[rgb]{0.53,0.00,0.00}{{#1}}}
    \newcommand{\SpecialCharTok}[1]{\textcolor[rgb]{0.25,0.44,0.63}{{#1}}}
    \newcommand{\VerbatimStringTok}[1]{\textcolor[rgb]{0.25,0.44,0.63}{{#1}}}
    \newcommand{\SpecialStringTok}[1]{\textcolor[rgb]{0.73,0.40,0.53}{{#1}}}
    \newcommand{\ImportTok}[1]{{#1}}
    \newcommand{\DocumentationTok}[1]{\textcolor[rgb]{0.73,0.13,0.13}{\textit{{#1}}}}
    \newcommand{\AnnotationTok}[1]{\textcolor[rgb]{0.38,0.63,0.69}{\textbf{\textit{{#1}}}}}
    \newcommand{\CommentVarTok}[1]{\textcolor[rgb]{0.38,0.63,0.69}{\textbf{\textit{{#1}}}}}
    \newcommand{\VariableTok}[1]{\textcolor[rgb]{0.10,0.09,0.49}{{#1}}}
    \newcommand{\ControlFlowTok}[1]{\textcolor[rgb]{0.00,0.44,0.13}{\textbf{{#1}}}}
    \newcommand{\OperatorTok}[1]{\textcolor[rgb]{0.40,0.40,0.40}{{#1}}}
    \newcommand{\BuiltInTok}[1]{{#1}}
    \newcommand{\ExtensionTok}[1]{{#1}}
    \newcommand{\PreprocessorTok}[1]{\textcolor[rgb]{0.74,0.48,0.00}{{#1}}}
    \newcommand{\AttributeTok}[1]{\textcolor[rgb]{0.49,0.56,0.16}{{#1}}}
    \newcommand{\InformationTok}[1]{\textcolor[rgb]{0.38,0.63,0.69}{\textbf{\textit{{#1}}}}}
    \newcommand{\WarningTok}[1]{\textcolor[rgb]{0.38,0.63,0.69}{\textbf{\textit{{#1}}}}}
    
    
    % Define a nice break command that doesn't care if a line doesn't already
    % exist.
    \def\br{\hspace*{\fill} \\* }
    % Math Jax compatability definitions
    \def\gt{>}
    \def\lt{<}
    % Document parameters
    \title{L3-Python-for-Statistical-Modeling}
    
    
    

    % Pygments definitions
    
\makeatletter
\def\PY@reset{\let\PY@it=\relax \let\PY@bf=\relax%
    \let\PY@ul=\relax \let\PY@tc=\relax%
    \let\PY@bc=\relax \let\PY@ff=\relax}
\def\PY@tok#1{\csname PY@tok@#1\endcsname}
\def\PY@toks#1+{\ifx\relax#1\empty\else%
    \PY@tok{#1}\expandafter\PY@toks\fi}
\def\PY@do#1{\PY@bc{\PY@tc{\PY@ul{%
    \PY@it{\PY@bf{\PY@ff{#1}}}}}}}
\def\PY#1#2{\PY@reset\PY@toks#1+\relax+\PY@do{#2}}

\expandafter\def\csname PY@tok@gd\endcsname{\def\PY@tc##1{\textcolor[rgb]{0.63,0.00,0.00}{##1}}}
\expandafter\def\csname PY@tok@gu\endcsname{\let\PY@bf=\textbf\def\PY@tc##1{\textcolor[rgb]{0.50,0.00,0.50}{##1}}}
\expandafter\def\csname PY@tok@gt\endcsname{\def\PY@tc##1{\textcolor[rgb]{0.00,0.27,0.87}{##1}}}
\expandafter\def\csname PY@tok@gs\endcsname{\let\PY@bf=\textbf}
\expandafter\def\csname PY@tok@gr\endcsname{\def\PY@tc##1{\textcolor[rgb]{1.00,0.00,0.00}{##1}}}
\expandafter\def\csname PY@tok@cm\endcsname{\let\PY@it=\textit\def\PY@tc##1{\textcolor[rgb]{0.25,0.50,0.50}{##1}}}
\expandafter\def\csname PY@tok@vg\endcsname{\def\PY@tc##1{\textcolor[rgb]{0.10,0.09,0.49}{##1}}}
\expandafter\def\csname PY@tok@vi\endcsname{\def\PY@tc##1{\textcolor[rgb]{0.10,0.09,0.49}{##1}}}
\expandafter\def\csname PY@tok@mh\endcsname{\def\PY@tc##1{\textcolor[rgb]{0.40,0.40,0.40}{##1}}}
\expandafter\def\csname PY@tok@cs\endcsname{\let\PY@it=\textit\def\PY@tc##1{\textcolor[rgb]{0.25,0.50,0.50}{##1}}}
\expandafter\def\csname PY@tok@ge\endcsname{\let\PY@it=\textit}
\expandafter\def\csname PY@tok@vc\endcsname{\def\PY@tc##1{\textcolor[rgb]{0.10,0.09,0.49}{##1}}}
\expandafter\def\csname PY@tok@il\endcsname{\def\PY@tc##1{\textcolor[rgb]{0.40,0.40,0.40}{##1}}}
\expandafter\def\csname PY@tok@go\endcsname{\def\PY@tc##1{\textcolor[rgb]{0.53,0.53,0.53}{##1}}}
\expandafter\def\csname PY@tok@cp\endcsname{\def\PY@tc##1{\textcolor[rgb]{0.74,0.48,0.00}{##1}}}
\expandafter\def\csname PY@tok@gi\endcsname{\def\PY@tc##1{\textcolor[rgb]{0.00,0.63,0.00}{##1}}}
\expandafter\def\csname PY@tok@gh\endcsname{\let\PY@bf=\textbf\def\PY@tc##1{\textcolor[rgb]{0.00,0.00,0.50}{##1}}}
\expandafter\def\csname PY@tok@ni\endcsname{\let\PY@bf=\textbf\def\PY@tc##1{\textcolor[rgb]{0.60,0.60,0.60}{##1}}}
\expandafter\def\csname PY@tok@nl\endcsname{\def\PY@tc##1{\textcolor[rgb]{0.63,0.63,0.00}{##1}}}
\expandafter\def\csname PY@tok@nn\endcsname{\let\PY@bf=\textbf\def\PY@tc##1{\textcolor[rgb]{0.00,0.00,1.00}{##1}}}
\expandafter\def\csname PY@tok@no\endcsname{\def\PY@tc##1{\textcolor[rgb]{0.53,0.00,0.00}{##1}}}
\expandafter\def\csname PY@tok@na\endcsname{\def\PY@tc##1{\textcolor[rgb]{0.49,0.56,0.16}{##1}}}
\expandafter\def\csname PY@tok@nb\endcsname{\def\PY@tc##1{\textcolor[rgb]{0.00,0.50,0.00}{##1}}}
\expandafter\def\csname PY@tok@nc\endcsname{\let\PY@bf=\textbf\def\PY@tc##1{\textcolor[rgb]{0.00,0.00,1.00}{##1}}}
\expandafter\def\csname PY@tok@nd\endcsname{\def\PY@tc##1{\textcolor[rgb]{0.67,0.13,1.00}{##1}}}
\expandafter\def\csname PY@tok@ne\endcsname{\let\PY@bf=\textbf\def\PY@tc##1{\textcolor[rgb]{0.82,0.25,0.23}{##1}}}
\expandafter\def\csname PY@tok@nf\endcsname{\def\PY@tc##1{\textcolor[rgb]{0.00,0.00,1.00}{##1}}}
\expandafter\def\csname PY@tok@si\endcsname{\let\PY@bf=\textbf\def\PY@tc##1{\textcolor[rgb]{0.73,0.40,0.53}{##1}}}
\expandafter\def\csname PY@tok@s2\endcsname{\def\PY@tc##1{\textcolor[rgb]{0.73,0.13,0.13}{##1}}}
\expandafter\def\csname PY@tok@nt\endcsname{\let\PY@bf=\textbf\def\PY@tc##1{\textcolor[rgb]{0.00,0.50,0.00}{##1}}}
\expandafter\def\csname PY@tok@nv\endcsname{\def\PY@tc##1{\textcolor[rgb]{0.10,0.09,0.49}{##1}}}
\expandafter\def\csname PY@tok@s1\endcsname{\def\PY@tc##1{\textcolor[rgb]{0.73,0.13,0.13}{##1}}}
\expandafter\def\csname PY@tok@ch\endcsname{\let\PY@it=\textit\def\PY@tc##1{\textcolor[rgb]{0.25,0.50,0.50}{##1}}}
\expandafter\def\csname PY@tok@m\endcsname{\def\PY@tc##1{\textcolor[rgb]{0.40,0.40,0.40}{##1}}}
\expandafter\def\csname PY@tok@gp\endcsname{\let\PY@bf=\textbf\def\PY@tc##1{\textcolor[rgb]{0.00,0.00,0.50}{##1}}}
\expandafter\def\csname PY@tok@sh\endcsname{\def\PY@tc##1{\textcolor[rgb]{0.73,0.13,0.13}{##1}}}
\expandafter\def\csname PY@tok@ow\endcsname{\let\PY@bf=\textbf\def\PY@tc##1{\textcolor[rgb]{0.67,0.13,1.00}{##1}}}
\expandafter\def\csname PY@tok@sx\endcsname{\def\PY@tc##1{\textcolor[rgb]{0.00,0.50,0.00}{##1}}}
\expandafter\def\csname PY@tok@bp\endcsname{\def\PY@tc##1{\textcolor[rgb]{0.00,0.50,0.00}{##1}}}
\expandafter\def\csname PY@tok@c1\endcsname{\let\PY@it=\textit\def\PY@tc##1{\textcolor[rgb]{0.25,0.50,0.50}{##1}}}
\expandafter\def\csname PY@tok@o\endcsname{\def\PY@tc##1{\textcolor[rgb]{0.40,0.40,0.40}{##1}}}
\expandafter\def\csname PY@tok@kc\endcsname{\let\PY@bf=\textbf\def\PY@tc##1{\textcolor[rgb]{0.00,0.50,0.00}{##1}}}
\expandafter\def\csname PY@tok@c\endcsname{\let\PY@it=\textit\def\PY@tc##1{\textcolor[rgb]{0.25,0.50,0.50}{##1}}}
\expandafter\def\csname PY@tok@mf\endcsname{\def\PY@tc##1{\textcolor[rgb]{0.40,0.40,0.40}{##1}}}
\expandafter\def\csname PY@tok@err\endcsname{\def\PY@bc##1{\setlength{\fboxsep}{0pt}\fcolorbox[rgb]{1.00,0.00,0.00}{1,1,1}{\strut ##1}}}
\expandafter\def\csname PY@tok@mb\endcsname{\def\PY@tc##1{\textcolor[rgb]{0.40,0.40,0.40}{##1}}}
\expandafter\def\csname PY@tok@ss\endcsname{\def\PY@tc##1{\textcolor[rgb]{0.10,0.09,0.49}{##1}}}
\expandafter\def\csname PY@tok@sr\endcsname{\def\PY@tc##1{\textcolor[rgb]{0.73,0.40,0.53}{##1}}}
\expandafter\def\csname PY@tok@mo\endcsname{\def\PY@tc##1{\textcolor[rgb]{0.40,0.40,0.40}{##1}}}
\expandafter\def\csname PY@tok@kd\endcsname{\let\PY@bf=\textbf\def\PY@tc##1{\textcolor[rgb]{0.00,0.50,0.00}{##1}}}
\expandafter\def\csname PY@tok@mi\endcsname{\def\PY@tc##1{\textcolor[rgb]{0.40,0.40,0.40}{##1}}}
\expandafter\def\csname PY@tok@kn\endcsname{\let\PY@bf=\textbf\def\PY@tc##1{\textcolor[rgb]{0.00,0.50,0.00}{##1}}}
\expandafter\def\csname PY@tok@cpf\endcsname{\let\PY@it=\textit\def\PY@tc##1{\textcolor[rgb]{0.25,0.50,0.50}{##1}}}
\expandafter\def\csname PY@tok@kr\endcsname{\let\PY@bf=\textbf\def\PY@tc##1{\textcolor[rgb]{0.00,0.50,0.00}{##1}}}
\expandafter\def\csname PY@tok@s\endcsname{\def\PY@tc##1{\textcolor[rgb]{0.73,0.13,0.13}{##1}}}
\expandafter\def\csname PY@tok@kp\endcsname{\def\PY@tc##1{\textcolor[rgb]{0.00,0.50,0.00}{##1}}}
\expandafter\def\csname PY@tok@w\endcsname{\def\PY@tc##1{\textcolor[rgb]{0.73,0.73,0.73}{##1}}}
\expandafter\def\csname PY@tok@kt\endcsname{\def\PY@tc##1{\textcolor[rgb]{0.69,0.00,0.25}{##1}}}
\expandafter\def\csname PY@tok@sc\endcsname{\def\PY@tc##1{\textcolor[rgb]{0.73,0.13,0.13}{##1}}}
\expandafter\def\csname PY@tok@sb\endcsname{\def\PY@tc##1{\textcolor[rgb]{0.73,0.13,0.13}{##1}}}
\expandafter\def\csname PY@tok@k\endcsname{\let\PY@bf=\textbf\def\PY@tc##1{\textcolor[rgb]{0.00,0.50,0.00}{##1}}}
\expandafter\def\csname PY@tok@se\endcsname{\let\PY@bf=\textbf\def\PY@tc##1{\textcolor[rgb]{0.73,0.40,0.13}{##1}}}
\expandafter\def\csname PY@tok@sd\endcsname{\let\PY@it=\textit\def\PY@tc##1{\textcolor[rgb]{0.73,0.13,0.13}{##1}}}

\def\PYZbs{\char`\\}
\def\PYZus{\char`\_}
\def\PYZob{\char`\{}
\def\PYZcb{\char`\}}
\def\PYZca{\char`\^}
\def\PYZam{\char`\&}
\def\PYZlt{\char`\<}
\def\PYZgt{\char`\>}
\def\PYZsh{\char`\#}
\def\PYZpc{\char`\%}
\def\PYZdl{\char`\$}
\def\PYZhy{\char`\-}
\def\PYZsq{\char`\'}
\def\PYZdq{\char`\"}
\def\PYZti{\char`\~}
% for compatibility with earlier versions
\def\PYZat{@}
\def\PYZlb{[}
\def\PYZrb{]}
\makeatother


    % Exact colors from NB
    \definecolor{incolor}{rgb}{0.0, 0.0, 0.5}
    \definecolor{outcolor}{rgb}{0.545, 0.0, 0.0}



    
    % Prevent overflowing lines due to hard-to-break entities
    \sloppy 
    % Setup hyperref package
    \hypersetup{
      breaklinks=true,  % so long urls are correctly broken across lines
      colorlinks=true,
      urlcolor=blue,
      linkcolor=darkorange,
      citecolor=darkgreen,
      }
    % Slightly bigger margins than the latex defaults
    
    \geometry{verbose,tmargin=1in,bmargin=1in,lmargin=1in,rmargin=1in}
    
    

    \begin{document}
    
    
    \maketitle
    
    

    
    \section{Python modules for Statistics
(Python统计模块)}\label{python-modules-for-statistics-pythonux7edfux8ba1ux6a21ux5757}

\subsection{NumPy}\label{numpy}

\texttt{NumPy} is short for Numerical Python, is the foundational
package for scientific computing in Python. It contains among other
things:

\begin{itemize}
\itemsep1pt\parskip0pt\parsep0pt
\item
  a powerful N-dimensional array object
\item
  sophisticated (broadcasting) functions
\item
  tools for integrating C/C++ and Fortran code
\item
  useful linear algebra, Fourier transform, and random number
  capabilities
\end{itemize}

Besides its obvious scientific uses, NumPy can also be used as an
efficient multi-dimensional container of generic data. Arbitrary
data-types can be defined. This allows NumPy to seamlessly and speedily
integrate with a wide variety of databases.

\begin{itemize}
\itemsep1pt\parskip0pt\parsep0pt
\item
  \href{http://docs.scipy.org/doc/numpy/reference/}{NumPy Reference}
\item
  \href{http://docs.scipy.org/doc/numpy/user/index.html}{NumPy User
  Guide}
\end{itemize}

\subsection{SciPy}\label{scipy}

\texttt{SciPy} is a collection of packages addressing a number of
different standard problem domains in scientific computing. Here is a
sampling of the packages included:

\begin{itemize}
\item
  \texttt{scipy.integrate} : numerical integration routines and
  differential equation solvers.
\item
  \texttt{scipy.linalg} : linear algebra routines and matrix
  decompositions extending beyond those provided in
  \texttt{numpy.linalg}.
\item
  \texttt{scipy.optimize} : function optimizers (minimizers) and root
  finding algorithms.
\item
  \texttt{scipy.signal} : signal processing tools.
\item
  \texttt{scipy.sparse} : sparse matrices and sparse linear system
  solvers.
\item
  \texttt{scipy.special} : wrapper around SPECFUN, a Fortran library
  implementing many common mathematical functions, such as the gamma
  function.
\item
  \texttt{scipy.stats} : standard continuous and discrete probability
  distributions (density functions, samplers, continuous distribution
  functions), various statistical tests, and more descriptive
  statistics.
\item
  \texttt{scipy.weave} : tool for using inline C++ code to accelerate
  array computations.
\item
  \texttt{scipy.cluster} : Clustering algorithms
\item
  \texttt{scipy.fftpack} : Fast Fourier Transform routines
\item
  \texttt{scipy.integrate} : Integration and ordinary differential
  equation solvers
\item
  \texttt{scipy.interpolate} : Interpolation and smoothing splines
\item
  \texttt{scipy.ndimage} : N-dimensional image processing optimize
  Optimization and root-finding routines
\item
  \texttt{scipy.spatial} : Spatial data structures and algorithms
\end{itemize}

\href{http://docs.scipy.org/doc/scipy/reference/}{SciPy Reference Guide}

\subsection{pandas}\label{pandas}

\texttt{pandas} provides rich data structures and functions designed to
make working with structured data fast, easy, and expressive. It is, as
you will see, one of the critical in-gredients enabling Python to be a
powerful and productive data analysis environment. pandas combines the
high performance array-computing features of \texttt{NumPy} with the
flexible data manipulation capabilities of spreadsheets and relational
databases (such as SQL). It provides sophisticated indexing
functionality to make it easy to reshape, slice and dice, perform
aggregations, and select subsets of data.

pandas consists of the following things

\begin{itemize}
\itemsep1pt\parskip0pt\parsep0pt
\item
  A set of labeled array data structures, the primary of which are
  Series and DataFrame
\item
  Index objects enabling both simple axis indexing and multi-level /
  hierarchical axis indexing
\item
  An integrated group by engine for aggregating and transforming data
  sets
\item
  Date range generation (date\_range) and custom date offsets enabling
  the implementation of customized frequencies
\item
  Input/Output tools: loading tabular data from flat files (CSV,
  delimited, Excel 2003), and saving and loading pandas objects from the
  fast and efficient PyTables/HDF5 format.
\item
  Memory-efficient ``sparse'' versions of the standard data structures
  for storing data that is mostly missing or mostly constant (some fixed
  value)
\item
  Moving window statistics (rolling mean, rolling standard deviation,
  etc.)
\item
  Static and moving window linear and panel regression
\end{itemize}

\href{http://pandas.pydata.org/pandas-docs/version/0.17.0/}{pandas
Documentation}

\subsection{matplotlib}\label{matplotlib}

\texttt{matplotlib} is the most popular Python library for producing
plots and other 2D data visualizations. It was originally created by
John D. Hunter (JDH) and is now maintained by a large team of
developers. It is well-suited for creating plots suitable for
publication. It integrates well with IPython, thus providing a
comfortable interactive environment for plotting and exploring data. The
plots are also interactive; you can zoom in on a section of the plot and
pan around the plot using the toolbar in the plot window.

\begin{itemize}
\itemsep1pt\parskip0pt\parsep0pt
\item
  \href{http://matplotlib.org/1.4.3/users/index.html}{matplotlib User
  Guide}
\item
  \href{http://matplotlib.org/1.4.3/gallery.html}{matplotlib Gallery}
\end{itemize}

    \section{Login to and send data to Linux server and vice versa
(访问远程Linux主机)}\label{login-to-and-send-data-to-linux-server-and-vice-versa-ux8bbfux95eeux8fdcux7a0blinuxux4e3bux673a}

Assume you have a Linux server you can login with host address
\texttt{11.22.33.44}, ssh port \texttt{22} (\texttt{22} is the default
port), user name \texttt{myusername} and password as \texttt{mysecret}

\subsection{If your local computer is Windows
(如果接入计算机是Windows)}\label{if-your-local-computer-is-windows-ux5982ux679cux63a5ux5165ux8ba1ux7b97ux673aux662fwindows}

\begin{itemize}
\item
  To login to a Linux server in Windows, download
  \href{http://www.chiark.greenend.org.uk/~sgtatham/putty/download.html}{Putty}
  and install it on your Windows machine. Follow the software's
  instructions to login.
\item
  To send and receive files, download
  \href{https://filezilla-project.org/}{FileZilla} and install it in you
  Windows machine. FileZilla works in Linux and Mac as well.
\item
  Start FileZilla and type \texttt{11.22.33.44} in \textbf{Host},
  \texttt{myusername} in \textbf{Username}, \texttt{mysecret} in
  \textbf{Password}, and \texttt{22} in \textbf{Port}. Now click
  \textbf{Quickconnect} to login to the server.
\item
  Then you can send and receive data by draging and dropping files from
  and to you server's home folder.
\end{itemize}

\subsection{If your local computer is Mac or Linux
(如果接入计算机是苹果或者Linux)}\label{if-your-local-computer-is-mac-or-linux-ux5982ux679cux63a5ux5165ux8ba1ux7b97ux673aux662fux82f9ux679cux6216ux8005linux}

\subsubsection{Login to server from your Mac or Linux
(从苹果电脑登陆到Linux服务器)}\label{login-to-server-from-your-mac-or-linux-ux4eceux82f9ux679cux7535ux8111ux767bux9646ux5230linuxux670dux52a1ux5668}

\begin{itemize}
\item
  Start a terminal on your local computer and open the file
  \texttt{\textasciitilde{}/.ssh/config} (create it if not exist)

\begin{verbatim}
emacs ~/.ssh/config
\end{verbatim}
\item
  Copy the following information to the file and save the file.

\begin{verbatim}
Host myserver1
     Hostname 11.22.33.44
     Port 22
     User myusername
\end{verbatim}
\item
  Now in your local computer's terminal, you can login to your server
  directly (answer \texttt{yes} to any prompt during your first login).

\begin{verbatim}
ssh myserver1
\end{verbatim}
\end{itemize}

\subsubsection{Send data to Linux server and vice versa from your Mac or
Linux
()}\label{send-data-to-linux-server-and-vice-versa-from-your-mac-or-linux}

\begin{itemize}
\item
  If you use Linux, check whether you have \texttt{rsync} installed on
  your local computer with \texttt{rsync -\/-version} in a terminal. If
  that does not exist, install it with
  \texttt{sudo apt-get install rsync}. Mac has rsync installed by
  default.
\item
  If you have a file called \texttt{stocks.csv} in your local computer's
  folder \texttt{\textasciitilde{}/Desktop/}, To send it to your linux
  server's folder \texttt{\textasciitilde{}/myproject/}, launch a
  terminal on your local computer, and type

\begin{verbatim}
rsync -av ~/Desktop/stocks myserver1:myproject/
\end{verbatim}
\item
  If you have a file called \texttt{stocks.csv} in your server's folder
  \texttt{\textasciitilde{}/myproject/}, To send it to your local
  computer's folder \texttt{\textasciitilde{}/Desktop/}, launch a
  terminal on your local computer, and type

\begin{verbatim}
rsync -av myserver1:myproject/stocks.csv ~/Desktop
\end{verbatim}
\item
  Type \texttt{man rsync} to see the complete manual of \texttt{rsync}.
\end{itemize}

    \section{Installing Python modules
(安装Python模块)}\label{installing-python-modules-ux5b89ux88c5pythonux6a21ux5757}

A lot of well-known packages are available in your Linux distribution.
If you want to install say e.g. \texttt{numpy} in Python 3, launch a
terminal and type in Debian/Ubuntu

\begin{verbatim}
    sudo apt-get install python3-numpy
\end{verbatim}

To install packages from PyPI (the Python Package Index), Please consult
the
\href{https://python-packaging-user-guide.readthedocs.org/en/latest/installing/}{Python
Packaging User Guide}.

    \section{Working with data
(Pyhton数据操作)}\label{working-with-data-pyhtonux6570ux636eux64cdux4f5c}

    \subsection{Read and write data in Python with \texttt{stdin} and
\texttt{stdout}
(利用标准输入数处读写数据)}\label{read-and-write-data-in-python-with-stdin-and-stdout-ux5229ux7528ux6807ux51c6ux8f93ux5165ux6570ux5904ux8bfbux5199ux6570ux636e}

    \begin{Verbatim}[commandchars=\\\{\}]
{\color{incolor}In [{\color{incolor}1}]:} \PY{c+ch}{\PYZsh{}! /usr/bin/env python3}
        \PY{c+c1}{\PYZsh{} line\PYZus{}count.py}
        \PY{k+kn}{import} \PY{n+nn}{sys}
        \PY{n}{count} \PY{o}{=} \PY{l+m+mi}{0}
        \PY{n}{data} \PY{o}{=} \PY{p}{[}\PY{p}{]}
        \PY{k}{for} \PY{n}{line} \PY{o+ow}{in} \PY{n}{sys}\PY{o}{.}\PY{n}{stdin}\PY{p}{:}
            \PY{n}{count} \PY{o}{+}\PY{o}{=} \PY{l+m+mi}{1}
            \PY{n}{data}\PY{o}{.}\PY{n}{append}\PY{p}{(}\PY{n}{line}\PY{p}{)}    
        \PY{n+nb}{print}\PY{p}{(}\PY{n}{count}\PY{p}{)} \PY{c+c1}{\PYZsh{} print goes to sys.stdout}
        \PY{n+nb}{print}\PY{p}{(}\PY{n}{data}\PY{p}{)}
\end{Verbatim}

    \begin{Verbatim}[commandchars=\\\{\}]
0
[]
    \end{Verbatim}

    Then launch a terminal and first make your Python script executable.
Then send you \texttt{testFile} to your Python script

\begin{verbatim}
chmod +x line_count.py
cat L3-Python-for-Statistical-Modeling.html | line_count.py
\end{verbatim}

    \subsection{Read from and write to files directly
(直接读解数据)}\label{read-from-and-write-to-files-directly-ux76f4ux63a5ux8bfbux89e3ux6570ux636e}

You can also explicitly read from and write to files directly in your
code. Python makes working with files pretty simple.

    \begin{itemize}
\item
  The first step to working with a text file is to obtain a file object
  using \texttt{open()}

  `r' means read-only

\begin{verbatim}
file_for_reading = open('reading_file.txt', 'r')
\end{verbatim}

  `w' is write -- will destroy the file if it already exists!

\begin{verbatim}
file_for_writing = open('writing_file.txt', 'w')
\end{verbatim}

  `a' is append -- for adding to the end of the file

\begin{verbatim}
file_for_appending = open('appending_file.txt', 'a')
\end{verbatim}
\item
  The second step is do something with the file.
\item
  Don't forget to close your files when you're done.

\begin{verbatim}
file_for_writing.close()
\end{verbatim}
\end{itemize}

\textbf{Note} Because it is easy to forget to close your files, you
should always use them in a \textbf{with} block, at the end of which
they will be closed automatically:

\begin{verbatim}
with open(filename,'r') as f:
    data = function_that_gets_data_from(f)
\end{verbatim}

    \begin{Verbatim}[commandchars=\\\{\}]
{\color{incolor}In [{\color{incolor}2}]:} \PY{c+ch}{\PYZsh{}! /usr/bin/env python3}
        \PY{c+c1}{\PYZsh{} hash\PYZus{}check.py}
        \PY{k+kn}{import} \PY{n+nn}{re}
        \PY{n}{starts\PYZus{}with\PYZus{}hash} \PY{o}{=} \PY{l+m+mi}{0}
        
        \PY{c+c1}{\PYZsh{} look at each line in the file use a regex to see if it starts with \PYZsq{}\PYZsh{}\PYZsq{} if it does, add 1}
        \PY{c+c1}{\PYZsh{} to the count.}
        
        \PY{k}{with} \PY{n+nb}{open}\PY{p}{(}\PY{l+s+s1}{\PYZsq{}}\PY{l+s+s1}{line\PYZus{}count.py}\PY{l+s+s1}{\PYZsq{}}\PY{p}{,}\PY{l+s+s1}{\PYZsq{}}\PY{l+s+s1}{r}\PY{l+s+s1}{\PYZsq{}}\PY{p}{)} \PY{k}{as} \PY{n}{file}\PY{p}{:}
            \PY{k}{for} \PY{n}{line} \PY{o+ow}{in} \PY{n}{file}\PY{p}{:}
                \PY{k}{if} \PY{n}{re}\PY{o}{.}\PY{n}{match}\PY{p}{(}\PY{l+s+s2}{\PYZdq{}}\PY{l+s+s2}{\PYZca{}\PYZsh{}}\PY{l+s+s2}{\PYZdq{}}\PY{p}{,}\PY{n}{line}\PY{p}{)}\PY{p}{:}
                    \PY{n}{starts\PYZus{}with\PYZus{}hash} \PY{o}{+}\PY{o}{=} \PY{l+m+mi}{1}
        \PY{n+nb}{print}\PY{p}{(}\PY{n}{starts\PYZus{}with\PYZus{}hash}\PY{p}{)}
\end{Verbatim}

    \begin{Verbatim}[commandchars=\\\{\}]
1
    \end{Verbatim}

    \subsection{Read a CSV file
(读取CSV文件)}\label{read-a-csv-file-ux8bfbux53d6csvux6587ux4ef6}

If your file has no headers (which means you probably want each row as a
list , and which places the burden on you to know what's in each
column), you can use \texttt{csv.reader()} in \texttt{csv} module to
iterate over the rows, each of which will be an appropriately split
list.

If your file has headers, you can either skip the header row (with an
initial call to \texttt{reader.next()}) or get each row as a
\texttt{dict} (with the headers as keys) by using
\texttt{csv.DictReader()} in \texttt{module}:

symbol date closing\_price AAPL 2015-01-23 112.98 AAPL 2015-01-22 112.4
AAPL 2015-01-21 109.55 AAPL 2015-01-20 108.72 AAPL 2015-01-16 105.99
AAPL 2015-01-15 106.82 AAPL 2015-01-14 109.8 AAPL 2015-01-13 110.22 AAPL
2015-01-12 109.25

    \begin{Verbatim}[commandchars=\\\{\}]
{\color{incolor}In [{\color{incolor}3}]:} \PY{c+ch}{\PYZsh{}! /usr/bin/env python3}
        
        \PY{k+kn}{import} \PY{n+nn}{csv}
        
        \PY{n}{data} \PY{o}{=} \PY{p}{\PYZob{}}\PY{l+s+s1}{\PYZsq{}}\PY{l+s+s1}{date}\PY{l+s+s1}{\PYZsq{}}\PY{p}{:}\PY{p}{[}\PY{p}{]}\PY{p}{,} \PY{l+s+s1}{\PYZsq{}}\PY{l+s+s1}{symbol}\PY{l+s+s1}{\PYZsq{}}\PY{p}{:}\PY{p}{[}\PY{p}{]}\PY{p}{,} \PY{l+s+s1}{\PYZsq{}}\PY{l+s+s1}{closing\PYZus{}price}\PY{l+s+s1}{\PYZsq{}} \PY{p}{:} \PY{p}{[}\PY{p}{]}\PY{p}{\PYZcb{}}
        \PY{k}{with} \PY{n+nb}{open}\PY{p}{(}\PY{l+s+s1}{\PYZsq{}}\PY{l+s+s1}{stocks.csv}\PY{l+s+s1}{\PYZsq{}}\PY{p}{,} \PY{l+s+s1}{\PYZsq{}}\PY{l+s+s1}{r}\PY{l+s+s1}{\PYZsq{}}\PY{p}{)} \PY{k}{as} \PY{n}{f}\PY{p}{:}
            \PY{n}{reader} \PY{o}{=} \PY{n}{csv}\PY{o}{.}\PY{n}{DictReader}\PY{p}{(}\PY{n}{f}\PY{p}{,} \PY{n}{delimiter}\PY{o}{=}\PY{l+s+s1}{\PYZsq{}}\PY{l+s+se}{\PYZbs{}t}\PY{l+s+s1}{\PYZsq{}}\PY{p}{)}
            \PY{k}{for} \PY{n}{row} \PY{o+ow}{in} \PY{n}{reader}\PY{p}{:}
                \PY{n}{data}\PY{p}{[}\PY{l+s+s1}{\PYZsq{}}\PY{l+s+s1}{date}\PY{l+s+s1}{\PYZsq{}}\PY{p}{]}\PY{o}{.}\PY{n}{append}\PY{p}{(}\PY{n}{row}\PY{p}{[}\PY{l+s+s2}{\PYZdq{}}\PY{l+s+s2}{date}\PY{l+s+s2}{\PYZdq{}}\PY{p}{]}\PY{p}{)}
                \PY{n}{data}\PY{p}{[}\PY{l+s+s1}{\PYZsq{}}\PY{l+s+s1}{symbol}\PY{l+s+s1}{\PYZsq{}}\PY{p}{]}\PY{o}{.}\PY{n}{append}\PY{p}{(}\PY{n}{row}\PY{p}{[}\PY{l+s+s2}{\PYZdq{}}\PY{l+s+s2}{symbol}\PY{l+s+s2}{\PYZdq{}}\PY{p}{]}\PY{p}{)}
                \PY{n}{data}\PY{p}{[}\PY{l+s+s1}{\PYZsq{}}\PY{l+s+s1}{closing\PYZus{}price}\PY{l+s+s1}{\PYZsq{}}\PY{p}{]}\PY{o}{.}\PY{n}{append}\PY{p}{(}\PY{n+nb}{float}\PY{p}{(}\PY{n}{row}\PY{p}{[}\PY{l+s+s2}{\PYZdq{}}\PY{l+s+s2}{closing\PYZus{}price}\PY{l+s+s2}{\PYZdq{}}\PY{p}{]}\PY{p}{)}\PY{p}{)}
\end{Verbatim}

    \begin{Verbatim}[commandchars=\\\{\}]
{\color{incolor}In [{\color{incolor}4}]:} \PY{n}{data}\PY{o}{.}\PY{n}{keys}\PY{p}{(}\PY{p}{)}
\end{Verbatim}

            \begin{Verbatim}[commandchars=\\\{\}]
{\color{outcolor}Out[{\color{outcolor}4}]:} dict\_keys(['symbol', 'date', 'closing\_price'])
\end{Verbatim}
        
    Alternatively, \texttt{pandas} provides \texttt{read\_csv()} function to
read csv files

    \begin{Verbatim}[commandchars=\\\{\}]
{\color{incolor}In [{\color{incolor}5}]:} \PY{c+ch}{\PYZsh{}! /usr/bin/env python3}
        
        \PY{k+kn}{import} \PY{n+nn}{pandas}
        
        \PY{n}{data2} \PY{o}{=} \PY{n}{pandas}\PY{o}{.}\PY{n}{read\PYZus{}csv}\PY{p}{(}\PY{l+s+s1}{\PYZsq{}}\PY{l+s+s1}{stocks.csv}\PY{l+s+s1}{\PYZsq{}}\PY{p}{,} \PY{n}{delimiter}\PY{o}{=}\PY{l+s+s1}{\PYZsq{}}\PY{l+s+se}{\PYZbs{}t}\PY{l+s+s1}{\PYZsq{}}\PY{p}{,}\PY{n}{header}\PY{o}{=}\PY{k+kc}{None}\PY{p}{)}
        \PY{n+nb}{print}\PY{p}{(}\PY{n+nb}{len}\PY{p}{(}\PY{n}{data2}\PY{p}{)}\PY{p}{)}
        \PY{n+nb}{print}\PY{p}{(}\PY{n+nb}{type}\PY{p}{(}\PY{n}{data2}\PY{p}{)}\PY{p}{)}
\end{Verbatim}

    \begin{Verbatim}[commandchars=\\\{\}]
16556
<class 'pandas.core.frame.DataFrame'>
    \end{Verbatim}

    The pandas I/O API is a set of top level \texttt{reader} functions
accessed like \texttt{read\_csv()} that generally return a pandas
object. These functions includes

\begin{verbatim}
read_excel
read_hdf
read_sql
read_json
read_msgpack (experimental)
read_html
read_gbq (experimental)
read_stata
read_sas
read_clipboard
read_pickle
\end{verbatim}

See \href{http://pandas.pydata.org/pandas-docs/stable/io.html}{pandas IO
tools} for detailed explanation.

    \section{Linear Algebra
(线性代数)}\label{linear-algebra-ux7ebfux6027ux4ee3ux6570}

Linear algebra can be done conveniently via \texttt{scipy.linalg}. When
SciPy is built using the optimized ATLAS LAPACK and BLAS libraries, it
has very fast linear algebra capabilities. If you dig deep enough, all
of the raw lapack and blas libraries are available for your use for even
more speed. In this section, some easier-to-use interfaces to these
routines are described.

All of these linear algebra routines expect an object that can be
converted into a 2-dimensional array. The output of these routines is
also a two-dimensional array.

    \subsection{Matrices and n-dimensional array
(矩阵和多维数组)}\label{matrices-and-n-dimensional-array-ux77e9ux9635ux548cux591aux7ef4ux6570ux7ec4}

    \begin{Verbatim}[commandchars=\\\{\}]
{\color{incolor}In [{\color{incolor}6}]:} \PY{k+kn}{import} \PY{n+nn}{numpy} \PY{k}{as} \PY{n+nn}{np}
        \PY{k+kn}{from} \PY{n+nn}{scipy} \PY{k}{import} \PY{n}{linalg}
        \PY{n}{A} \PY{o}{=} \PY{n}{np}\PY{o}{.}\PY{n}{array}\PY{p}{(}\PY{p}{[}\PY{p}{[}\PY{l+m+mi}{1}\PY{p}{,}\PY{l+m+mi}{2}\PY{p}{]}\PY{p}{,}\PY{p}{[}\PY{l+m+mi}{3}\PY{p}{,}\PY{l+m+mi}{4}\PY{p}{]}\PY{p}{]}\PY{p}{)}
        \PY{n}{A}
\end{Verbatim}

            \begin{Verbatim}[commandchars=\\\{\}]
{\color{outcolor}Out[{\color{outcolor}6}]:} array([[1, 2],
               [3, 4]])
\end{Verbatim}
        
    \begin{Verbatim}[commandchars=\\\{\}]
{\color{incolor}In [{\color{incolor}7}]:} \PY{n}{linalg}\PY{o}{.}\PY{n}{inv}\PY{p}{(}\PY{n}{A}\PY{p}{)} \PY{c+c1}{\PYZsh{} inverse of a matrix}
\end{Verbatim}

            \begin{Verbatim}[commandchars=\\\{\}]
{\color{outcolor}Out[{\color{outcolor}7}]:} array([[-2. ,  1. ],
               [ 1.5, -0.5]])
\end{Verbatim}
        
    \begin{Verbatim}[commandchars=\\\{\}]
{\color{incolor}In [{\color{incolor}8}]:} \PY{n}{b} \PY{o}{=} \PY{n}{np}\PY{o}{.}\PY{n}{array}\PY{p}{(}\PY{p}{[}\PY{p}{[}\PY{l+m+mi}{5}\PY{p}{,}\PY{l+m+mi}{6}\PY{p}{]}\PY{p}{]}\PY{p}{)} \PY{c+c1}{\PYZsh{}2D array}
        \PY{n}{b}
\end{Verbatim}

            \begin{Verbatim}[commandchars=\\\{\}]
{\color{outcolor}Out[{\color{outcolor}8}]:} array([[5, 6]])
\end{Verbatim}
        
    \begin{Verbatim}[commandchars=\\\{\}]
{\color{incolor}In [{\color{incolor}9}]:} \PY{n}{b}\PY{o}{.}\PY{n}{T}
\end{Verbatim}

            \begin{Verbatim}[commandchars=\\\{\}]
{\color{outcolor}Out[{\color{outcolor}9}]:} array([[5],
               [6]])
\end{Verbatim}
        
    \begin{Verbatim}[commandchars=\\\{\}]
{\color{incolor}In [{\color{incolor}10}]:} \PY{n}{A}\PY{o}{*}\PY{n}{b} \PY{c+c1}{\PYZsh{}not matrix multiplication!}
\end{Verbatim}

            \begin{Verbatim}[commandchars=\\\{\}]
{\color{outcolor}Out[{\color{outcolor}10}]:} array([[ 5, 12],
                [15, 24]])
\end{Verbatim}
        
    \begin{Verbatim}[commandchars=\\\{\}]
{\color{incolor}In [{\color{incolor}11}]:} \PY{n}{A}\PY{o}{.}\PY{n}{dot}\PY{p}{(}\PY{n}{b}\PY{o}{.}\PY{n}{T}\PY{p}{)} \PY{c+c1}{\PYZsh{}matrix multiplication}
\end{Verbatim}

            \begin{Verbatim}[commandchars=\\\{\}]
{\color{outcolor}Out[{\color{outcolor}11}]:} array([[17],
                [39]])
\end{Verbatim}
        
    \begin{Verbatim}[commandchars=\\\{\}]
{\color{incolor}In [{\color{incolor}12}]:} \PY{n}{b} \PY{o}{=} \PY{n}{np}\PY{o}{.}\PY{n}{array}\PY{p}{(}\PY{p}{[}\PY{l+m+mi}{5}\PY{p}{,}\PY{l+m+mi}{6}\PY{p}{]}\PY{p}{)} \PY{c+c1}{\PYZsh{}1D array}
         \PY{n}{b}
\end{Verbatim}

            \begin{Verbatim}[commandchars=\\\{\}]
{\color{outcolor}Out[{\color{outcolor}12}]:} array([5, 6])
\end{Verbatim}
        
    \begin{Verbatim}[commandchars=\\\{\}]
{\color{incolor}In [{\color{incolor}13}]:} \PY{n}{b}\PY{o}{.}\PY{n}{T}  \PY{c+c1}{\PYZsh{}not matrix transpose!}
\end{Verbatim}

            \begin{Verbatim}[commandchars=\\\{\}]
{\color{outcolor}Out[{\color{outcolor}13}]:} array([5, 6])
\end{Verbatim}
        
    \begin{Verbatim}[commandchars=\\\{\}]
{\color{incolor}In [{\color{incolor}14}]:} \PY{n}{A}\PY{o}{.}\PY{n}{dot}\PY{p}{(}\PY{n}{b}\PY{p}{)}  \PY{c+c1}{\PYZsh{}does not matter for multiplication}
\end{Verbatim}

            \begin{Verbatim}[commandchars=\\\{\}]
{\color{outcolor}Out[{\color{outcolor}14}]:} array([17, 39])
\end{Verbatim}
        
    \subsection{Solving linear system
(求解线性方程组)}\label{solving-linear-system-ux6c42ux89e3ux7ebfux6027ux65b9ux7a0bux7ec4}

    \begin{Verbatim}[commandchars=\\\{\}]
{\color{incolor}In [{\color{incolor}15}]:} \PY{k+kn}{import} \PY{n+nn}{numpy} \PY{k}{as} \PY{n+nn}{np}
         \PY{k+kn}{from} \PY{n+nn}{scipy} \PY{k}{import} \PY{n}{linalg}
         \PY{n}{A} \PY{o}{=} \PY{n}{np}\PY{o}{.}\PY{n}{array}\PY{p}{(}\PY{p}{[}\PY{p}{[}\PY{l+m+mi}{1}\PY{p}{,}\PY{l+m+mi}{2}\PY{p}{]}\PY{p}{,}\PY{p}{[}\PY{l+m+mi}{3}\PY{p}{,}\PY{l+m+mi}{4}\PY{p}{]}\PY{p}{]}\PY{p}{)}
         \PY{n}{A}
\end{Verbatim}

            \begin{Verbatim}[commandchars=\\\{\}]
{\color{outcolor}Out[{\color{outcolor}15}]:} array([[1, 2],
                [3, 4]])
\end{Verbatim}
        
    \begin{Verbatim}[commandchars=\\\{\}]
{\color{incolor}In [{\color{incolor}16}]:} \PY{n}{b} \PY{o}{=} \PY{n}{np}\PY{o}{.}\PY{n}{array}\PY{p}{(}\PY{p}{[}\PY{p}{[}\PY{l+m+mi}{5}\PY{p}{]}\PY{p}{,}\PY{p}{[}\PY{l+m+mi}{6}\PY{p}{]}\PY{p}{]}\PY{p}{)}
         \PY{n}{b}
\end{Verbatim}

            \begin{Verbatim}[commandchars=\\\{\}]
{\color{outcolor}Out[{\color{outcolor}16}]:} array([[5],
                [6]])
\end{Verbatim}
        
    \begin{Verbatim}[commandchars=\\\{\}]
{\color{incolor}In [{\color{incolor}17}]:} \PY{n}{linalg}\PY{o}{.}\PY{n}{inv}\PY{p}{(}\PY{n}{A}\PY{p}{)}\PY{o}{.}\PY{n}{dot}\PY{p}{(}\PY{n}{b}\PY{p}{)} \PY{c+c1}{\PYZsh{}slow}
\end{Verbatim}

            \begin{Verbatim}[commandchars=\\\{\}]
{\color{outcolor}Out[{\color{outcolor}17}]:} array([[-4. ],
                [ 4.5]])
\end{Verbatim}
        
    \begin{Verbatim}[commandchars=\\\{\}]
{\color{incolor}In [{\color{incolor}18}]:} \PY{n}{A}\PY{o}{.}\PY{n}{dot}\PY{p}{(}\PY{n}{linalg}\PY{o}{.}\PY{n}{inv}\PY{p}{(}\PY{n}{A}\PY{p}{)}\PY{o}{.}\PY{n}{dot}\PY{p}{(}\PY{n}{b}\PY{p}{)}\PY{p}{)}\PY{o}{\PYZhy{}}\PY{n}{b} \PY{c+c1}{\PYZsh{}check}
\end{Verbatim}

            \begin{Verbatim}[commandchars=\\\{\}]
{\color{outcolor}Out[{\color{outcolor}18}]:} array([[ 0.],
                [ 0.]])
\end{Verbatim}
        
    \begin{Verbatim}[commandchars=\\\{\}]
{\color{incolor}In [{\color{incolor}19}]:} \PY{n}{np}\PY{o}{.}\PY{n}{linalg}\PY{o}{.}\PY{n}{solve}\PY{p}{(}\PY{n}{A}\PY{p}{,}\PY{n}{b}\PY{p}{)} \PY{c+c1}{\PYZsh{}fast}
\end{Verbatim}

            \begin{Verbatim}[commandchars=\\\{\}]
{\color{outcolor}Out[{\color{outcolor}19}]:} array([[-4. ],
                [ 4.5]])
\end{Verbatim}
        
    \begin{Verbatim}[commandchars=\\\{\}]
{\color{incolor}In [{\color{incolor}20}]:} \PY{n}{A}\PY{o}{.}\PY{n}{dot}\PY{p}{(}\PY{n}{np}\PY{o}{.}\PY{n}{linalg}\PY{o}{.}\PY{n}{solve}\PY{p}{(}\PY{n}{A}\PY{p}{,}\PY{n}{b}\PY{p}{)}\PY{p}{)}\PY{o}{\PYZhy{}}\PY{n}{b} \PY{c+c1}{\PYZsh{}check}
\end{Verbatim}

            \begin{Verbatim}[commandchars=\\\{\}]
{\color{outcolor}Out[{\color{outcolor}20}]:} array([[ 0.],
                [ 0.]])
\end{Verbatim}
        
    \subsection{Determinant
(行列式)}\label{determinant-ux884cux5217ux5f0f}

    \begin{Verbatim}[commandchars=\\\{\}]
{\color{incolor}In [{\color{incolor}21}]:} \PY{k+kn}{import} \PY{n+nn}{numpy} \PY{k}{as} \PY{n+nn}{np}
         \PY{k+kn}{from} \PY{n+nn}{scipy} \PY{k}{import} \PY{n}{linalg}
         \PY{n}{A} \PY{o}{=} \PY{n}{np}\PY{o}{.}\PY{n}{array}\PY{p}{(}\PY{p}{[}\PY{p}{[}\PY{l+m+mi}{1}\PY{p}{,}\PY{l+m+mi}{2}\PY{p}{]}\PY{p}{,}\PY{p}{[}\PY{l+m+mi}{3}\PY{p}{,}\PY{l+m+mi}{4}\PY{p}{]}\PY{p}{]}\PY{p}{)}
         \PY{n}{linalg}\PY{o}{.}\PY{n}{det}\PY{p}{(}\PY{n}{A}\PY{p}{)}
\end{Verbatim}

            \begin{Verbatim}[commandchars=\\\{\}]
{\color{outcolor}Out[{\color{outcolor}21}]:} -2.0
\end{Verbatim}
        
    \subsection{Least-squares problems and pseudo-inverses
(最小二乘和广义逆)}\label{least-squares-problems-and-pseudo-inverses-ux6700ux5c0fux4e8cux4e58ux548cux5e7fux4e49ux9006}

    \begin{Verbatim}[commandchars=\\\{\}]
{\color{incolor}In [{\color{incolor}22}]:} \PY{k+kn}{import} \PY{n+nn}{numpy} \PY{k}{as} \PY{n+nn}{np}
         \PY{k+kn}{from} \PY{n+nn}{scipy} \PY{k}{import} \PY{n}{linalg}
         \PY{k+kn}{import} \PY{n+nn}{matplotlib}\PY{n+nn}{.}\PY{n+nn}{pyplot} \PY{k}{as} \PY{n+nn}{plt}
\end{Verbatim}

    \begin{Verbatim}[commandchars=\\\{\}]
{\color{incolor}In [{\color{incolor}23}]:} \PY{n}{c1}\PY{p}{,} \PY{n}{c2} \PY{o}{=} \PY{l+m+mf}{5.0}\PY{p}{,} \PY{l+m+mf}{2.0}
         \PY{n}{i} \PY{o}{=} \PY{n}{np}\PY{o}{.}\PY{n}{r\PYZus{}}\PY{p}{[}\PY{l+m+mi}{1}\PY{p}{:}\PY{l+m+mi}{11}\PY{p}{]}
         \PY{n}{xi} \PY{o}{=} \PY{l+m+mf}{0.1}\PY{o}{*}\PY{n}{i}
         \PY{n}{yi} \PY{o}{=} \PY{n}{c1}\PY{o}{*}\PY{n}{np}\PY{o}{.}\PY{n}{exp}\PY{p}{(}\PY{o}{\PYZhy{}}\PY{n}{xi}\PY{p}{)} \PY{o}{+} \PY{n}{c2}\PY{o}{*}\PY{n}{xi}
         \PY{n}{zi} \PY{o}{=} \PY{n}{yi} \PY{o}{+} \PY{l+m+mf}{0.05} \PY{o}{*} \PY{n}{np}\PY{o}{.}\PY{n}{max}\PY{p}{(}\PY{n}{yi}\PY{p}{)} \PY{o}{*} \PY{n}{np}\PY{o}{.}\PY{n}{random}\PY{o}{.}\PY{n}{randn}\PY{p}{(}\PY{n+nb}{len}\PY{p}{(}\PY{n}{yi}\PY{p}{)}\PY{p}{)}
\end{Verbatim}

    \begin{Verbatim}[commandchars=\\\{\}]
{\color{incolor}In [{\color{incolor}24}]:} \PY{n}{A} \PY{o}{=} \PY{n}{np}\PY{o}{.}\PY{n}{c\PYZus{}}\PY{p}{[}\PY{n}{np}\PY{o}{.}\PY{n}{exp}\PY{p}{(}\PY{o}{\PYZhy{}}\PY{n}{xi}\PY{p}{)}\PY{p}{[}\PY{p}{:}\PY{p}{,} \PY{n}{np}\PY{o}{.}\PY{n}{newaxis}\PY{p}{]}\PY{p}{,} \PY{n}{xi}\PY{p}{[}\PY{p}{:}\PY{p}{,} \PY{n}{np}\PY{o}{.}\PY{n}{newaxis}\PY{p}{]}\PY{p}{]}
         \PY{n}{c}\PY{p}{,} \PY{n}{resid}\PY{p}{,} \PY{n}{rank}\PY{p}{,} \PY{n}{sigma} \PY{o}{=} \PY{n}{linalg}\PY{o}{.}\PY{n}{lstsq}\PY{p}{(}\PY{n}{A}\PY{p}{,} \PY{n}{zi}\PY{p}{)}
\end{Verbatim}

    \begin{Verbatim}[commandchars=\\\{\}]
{\color{incolor}In [{\color{incolor}25}]:} \PY{n}{xi2} \PY{o}{=} \PY{n}{np}\PY{o}{.}\PY{n}{r\PYZus{}}\PY{p}{[}\PY{l+m+mf}{0.1}\PY{p}{:}\PY{l+m+mf}{1.0}\PY{p}{:}\PY{l+m+mi}{100}\PY{n}{j}\PY{p}{]}
         \PY{n}{yi2} \PY{o}{=} \PY{n}{c}\PY{p}{[}\PY{l+m+mi}{0}\PY{p}{]}\PY{o}{*}\PY{n}{np}\PY{o}{.}\PY{n}{exp}\PY{p}{(}\PY{o}{\PYZhy{}}\PY{n}{xi2}\PY{p}{)} \PY{o}{+} \PY{n}{c}\PY{p}{[}\PY{l+m+mi}{1}\PY{p}{]}\PY{o}{*}\PY{n}{xi2}
\end{Verbatim}

    \begin{Verbatim}[commandchars=\\\{\}]
{\color{incolor}In [{\color{incolor}26}]:} \PY{n}{plt}\PY{o}{.}\PY{n}{plot}\PY{p}{(}\PY{n}{xi}\PY{p}{,}\PY{n}{zi}\PY{p}{,}\PY{l+s+s1}{\PYZsq{}}\PY{l+s+s1}{x}\PY{l+s+s1}{\PYZsq{}}\PY{p}{,}\PY{n}{xi2}\PY{p}{,}\PY{n}{yi2}\PY{p}{)}
         \PY{n}{plt}\PY{o}{.}\PY{n}{axis}\PY{p}{(}\PY{p}{[}\PY{l+m+mi}{0}\PY{p}{,}\PY{l+m+mf}{1.1}\PY{p}{,}\PY{l+m+mf}{3.0}\PY{p}{,}\PY{l+m+mf}{5.5}\PY{p}{]}\PY{p}{)}
         \PY{n}{plt}\PY{o}{.}\PY{n}{xlabel}\PY{p}{(}\PY{l+s+s1}{\PYZsq{}}\PY{l+s+s1}{\PYZdl{}x\PYZus{}i\PYZdl{}}\PY{l+s+s1}{\PYZsq{}}\PY{p}{)}
         \PY{n}{plt}\PY{o}{.}\PY{n}{title}\PY{p}{(}\PY{l+s+s1}{\PYZsq{}}\PY{l+s+s1}{Data fitting with linalg.lstsq}\PY{l+s+s1}{\PYZsq{}}\PY{p}{)}
         \PY{n}{plt}\PY{o}{.}\PY{n}{show}\PY{p}{(}\PY{p}{)}
\end{Verbatim}

    \subsection{Eigenvalues and eigenvectors
(特征向量和特征值)}\label{eigenvalues-and-eigenvectors-ux7279ux5f81ux5411ux91cfux548cux7279ux5f81ux503c}

    \begin{Verbatim}[commandchars=\\\{\}]
{\color{incolor}In [{\color{incolor}27}]:} \PY{k+kn}{import} \PY{n+nn}{numpy} \PY{k}{as} \PY{n+nn}{np}
         \PY{k+kn}{from} \PY{n+nn}{scipy} \PY{k}{import} \PY{n}{linalg}
         \PY{n}{A} \PY{o}{=} \PY{n}{np}\PY{o}{.}\PY{n}{array}\PY{p}{(}\PY{p}{[}\PY{p}{[}\PY{l+m+mi}{1}\PY{p}{,}\PY{l+m+mi}{2}\PY{p}{]}\PY{p}{,}\PY{p}{[}\PY{l+m+mi}{3}\PY{p}{,}\PY{l+m+mi}{4}\PY{p}{]}\PY{p}{]}\PY{p}{)}
         \PY{n}{la}\PY{p}{,}\PY{n}{v} \PY{o}{=} \PY{n}{linalg}\PY{o}{.}\PY{n}{eig}\PY{p}{(}\PY{n}{A}\PY{p}{)}
         \PY{n}{l1}\PY{p}{,}\PY{n}{l2} \PY{o}{=} \PY{n}{la}
         \PY{n+nb}{print}\PY{p}{(}\PY{n}{l1}\PY{p}{,} \PY{n}{l2}\PY{p}{)}  \PY{c+c1}{\PYZsh{}eigenvalues}
         
         \PY{n+nb}{print}\PY{p}{(}\PY{n}{v}\PY{p}{[}\PY{p}{:}\PY{p}{,}\PY{l+m+mi}{0}\PY{p}{]}\PY{p}{)}  \PY{c+c1}{\PYZsh{}first eigenvector}
         
         \PY{n+nb}{print}\PY{p}{(}\PY{n}{v}\PY{p}{[}\PY{p}{:}\PY{p}{,}\PY{l+m+mi}{1}\PY{p}{]}\PY{p}{)}  \PY{c+c1}{\PYZsh{}second eigenvector}
         
         \PY{n+nb}{print}\PY{p}{(}\PY{n}{np}\PY{o}{.}\PY{n}{sum}\PY{p}{(}\PY{n+nb}{abs}\PY{p}{(}\PY{n}{v}\PY{o}{*}\PY{o}{*}\PY{l+m+mi}{2}\PY{p}{)}\PY{p}{,}\PY{n}{axis}\PY{o}{=}\PY{l+m+mi}{0}\PY{p}{)}\PY{p}{)} \PY{c+c1}{\PYZsh{}eigenvectors are unitary}
         
         \PY{n}{v1} \PY{o}{=} \PY{n}{np}\PY{o}{.}\PY{n}{array}\PY{p}{(}\PY{n}{v}\PY{p}{[}\PY{p}{:}\PY{p}{,}\PY{l+m+mi}{0}\PY{p}{]}\PY{p}{)}\PY{o}{.}\PY{n}{T}
         \PY{n+nb}{print}\PY{p}{(}\PY{n}{linalg}\PY{o}{.}\PY{n}{norm}\PY{p}{(}\PY{n}{A}\PY{o}{.}\PY{n}{dot}\PY{p}{(}\PY{n}{v1}\PY{p}{)}\PY{o}{\PYZhy{}}\PY{n}{l1}\PY{o}{*}\PY{n}{v1}\PY{p}{)}\PY{p}{)} \PY{c+c1}{\PYZsh{}check the computation}
\end{Verbatim}

    \begin{Verbatim}[commandchars=\\\{\}]
(-0.372281323269+0j) (5.37228132327+0j)
[-0.82456484  0.56576746]
[-0.41597356 -0.90937671]
[ 1.  1.]
5.551115123125783e-17
    \end{Verbatim}

    \subsection{Singular Value Decomposition (SVD)
(奇异值分解)}\label{singular-value-decomposition-svd-ux5947ux5f02ux503cux5206ux89e3}

    \begin{Verbatim}[commandchars=\\\{\}]
{\color{incolor}In [{\color{incolor}28}]:} \PY{k+kn}{import} \PY{n+nn}{numpy} \PY{k}{as} \PY{n+nn}{np}
         \PY{k+kn}{from} \PY{n+nn}{scipy} \PY{k}{import} \PY{n}{linalg}
         \PY{n}{A} \PY{o}{=} \PY{n}{np}\PY{o}{.}\PY{n}{array}\PY{p}{(}\PY{p}{[}\PY{p}{[}\PY{l+m+mi}{1}\PY{p}{,}\PY{l+m+mi}{2}\PY{p}{,}\PY{l+m+mi}{3}\PY{p}{]}\PY{p}{,}\PY{p}{[}\PY{l+m+mi}{4}\PY{p}{,}\PY{l+m+mi}{5}\PY{p}{,}\PY{l+m+mi}{6}\PY{p}{]}\PY{p}{]}\PY{p}{)}
\end{Verbatim}

    \begin{Verbatim}[commandchars=\\\{\}]
{\color{incolor}In [{\color{incolor}29}]:} \PY{n}{M}\PY{p}{,}\PY{n}{N} \PY{o}{=} \PY{n}{A}\PY{o}{.}\PY{n}{shape}
         \PY{n}{U}\PY{p}{,}\PY{n}{s}\PY{p}{,}\PY{n}{Vh} \PY{o}{=} \PY{n}{linalg}\PY{o}{.}\PY{n}{svd}\PY{p}{(}\PY{n}{A}\PY{p}{)}
         \PY{n}{Sig} \PY{o}{=} \PY{n}{linalg}\PY{o}{.}\PY{n}{diagsvd}\PY{p}{(}\PY{n}{s}\PY{p}{,}\PY{n}{M}\PY{p}{,}\PY{n}{N}\PY{p}{)}
\end{Verbatim}

    \begin{Verbatim}[commandchars=\\\{\}]
{\color{incolor}In [{\color{incolor}30}]:} \PY{n}{U}\PY{p}{,} \PY{n}{Vh} \PY{o}{=} \PY{n}{U}\PY{p}{,} \PY{n}{Vh}
         \PY{n}{U}
\end{Verbatim}

            \begin{Verbatim}[commandchars=\\\{\}]
{\color{outcolor}Out[{\color{outcolor}30}]:} array([[-0.3863177 , -0.92236578],
                [-0.92236578,  0.3863177 ]])
\end{Verbatim}
        
    \begin{Verbatim}[commandchars=\\\{\}]
{\color{incolor}In [{\color{incolor}31}]:} \PY{n}{Sig}
\end{Verbatim}

            \begin{Verbatim}[commandchars=\\\{\}]
{\color{outcolor}Out[{\color{outcolor}31}]:} array([[ 9.508032  ,  0.        ,  0.        ],
                [ 0.        ,  0.77286964,  0.        ]])
\end{Verbatim}
        
    \begin{Verbatim}[commandchars=\\\{\}]
{\color{incolor}In [{\color{incolor}32}]:} \PY{n}{Vh}
\end{Verbatim}

            \begin{Verbatim}[commandchars=\\\{\}]
{\color{outcolor}Out[{\color{outcolor}32}]:} array([[-0.42866713, -0.56630692, -0.7039467 ],
                [ 0.80596391,  0.11238241, -0.58119908],
                [ 0.40824829, -0.81649658,  0.40824829]])
\end{Verbatim}
        
    \begin{Verbatim}[commandchars=\\\{\}]
{\color{incolor}In [{\color{incolor}33}]:} \PY{n}{U}\PY{o}{.}\PY{n}{dot}\PY{p}{(}\PY{n}{Sig}\PY{o}{.}\PY{n}{dot}\PY{p}{(}\PY{n}{Vh}\PY{p}{)}\PY{p}{)} \PY{c+c1}{\PYZsh{}check computation}
\end{Verbatim}

            \begin{Verbatim}[commandchars=\\\{\}]
{\color{outcolor}Out[{\color{outcolor}33}]:} array([[ 1.,  2.,  3.],
                [ 4.,  5.,  6.]])
\end{Verbatim}
        
    \subsection{QR decomposition
(QR分解)}\label{qr-decomposition-qrux5206ux89e3}

The command for QR decomposition is \texttt{linalg.qr}.

    \subsection{LU decomposition
(LU分解)}\label{lu-decomposition-luux5206ux89e3}

The SciPy command for this decomposition is \texttt{linalg.lu}. If the
intent for performing LU decomposition is for solving linear systems
then the command \texttt{linalg.lu\_factor} should be used followed by
repeated applications of the command \texttt{linalg.lu\_solve} to solve
the system for each new right-hand-side.

    \subsection{Cholesky decomposition
(乔列斯基分解)}\label{cholesky-decomposition-ux4e54ux5217ux65afux57faux5206ux89e3}

The command \texttt{linalg.cholesky} computes the cholesky
factorization. For using Cholesky factorization to solve systems of
equations there are also \texttt{linalg.cho\_factor} and
\texttt{linalg.cho\_solve} routines that work similarly to their LU
decomposition counterparts.

    \section{Statistical Distributions
(统计分布函数)}\label{statistical-distributions-ux7edfux8ba1ux5206ux5e03ux51fdux6570}

A large number of probability distributions as well as a growing library
of statistical functions are available in \texttt{scipy.stats}. See
http://docs.scipy.org/doc/scipy/reference/stats.html for a complete
list.

    Generate random numbers from normal distribution:

    \begin{Verbatim}[commandchars=\\\{\}]
{\color{incolor}In [{\color{incolor}34}]:} \PY{k+kn}{from} \PY{n+nn}{scipy}\PY{n+nn}{.}\PY{n+nn}{stats} \PY{k}{import} \PY{n}{norm}
         \PY{n}{r} \PY{o}{=} \PY{n}{norm}\PY{o}{.}\PY{n}{rvs}\PY{p}{(}\PY{n}{loc}\PY{o}{=}\PY{l+m+mi}{0}\PY{p}{,} \PY{n}{scale}\PY{o}{=}\PY{l+m+mi}{1}\PY{p}{,} \PY{n}{size}\PY{o}{=}\PY{l+m+mi}{1000}\PY{p}{)}
\end{Verbatim}

    Calculate a few first moments:

    \begin{Verbatim}[commandchars=\\\{\}]
{\color{incolor}In [{\color{incolor}35}]:} \PY{n}{mean}\PY{p}{,} \PY{n}{var}\PY{p}{,} \PY{n}{skew}\PY{p}{,} \PY{n}{kurt} \PY{o}{=} \PY{n}{norm}\PY{o}{.}\PY{n}{stats}\PY{p}{(}\PY{n}{moments}\PY{o}{=}\PY{l+s+s1}{\PYZsq{}}\PY{l+s+s1}{mvsk}\PY{l+s+s1}{\PYZsq{}}\PY{p}{)}
\end{Verbatim}

    Display the probability density function (pdf)

    \begin{Verbatim}[commandchars=\\\{\}]
{\color{incolor}In [{\color{incolor}36}]:} \PY{k+kn}{import} \PY{n+nn}{numpy} \PY{k}{as} \PY{n+nn}{np}
         \PY{k+kn}{import} \PY{n+nn}{matplotlib}\PY{n+nn}{.}\PY{n+nn}{pyplot} \PY{k}{as} \PY{n+nn}{plt}
         \PY{n}{x} \PY{o}{=} \PY{n}{np}\PY{o}{.}\PY{n}{linspace}\PY{p}{(}\PY{n}{norm}\PY{o}{.}\PY{n}{ppf}\PY{p}{(}\PY{l+m+mf}{0.01}\PY{p}{)}\PY{p}{,} \PY{c+c1}{\PYZsh{}ppf stands for percentiles.}
                         \PY{n}{norm}\PY{o}{.}\PY{n}{ppf}\PY{p}{(}\PY{l+m+mf}{0.99}\PY{p}{)}\PY{p}{,} \PY{l+m+mi}{100}\PY{p}{)}
         
         \PY{n}{fig}\PY{p}{,} \PY{n}{ax} \PY{o}{=} \PY{n}{plt}\PY{o}{.}\PY{n}{subplots}\PY{p}{(}\PY{l+m+mi}{1}\PY{p}{,} \PY{l+m+mi}{1}\PY{p}{)}
         \PY{n}{ax}\PY{o}{.}\PY{n}{plot}\PY{p}{(}\PY{n}{x}\PY{p}{,} \PY{n}{norm}\PY{o}{.}\PY{n}{pdf}\PY{p}{(}\PY{n}{x}\PY{p}{)}\PY{p}{,}
                 \PY{l+s+s1}{\PYZsq{}}\PY{l+s+s1}{r\PYZhy{}}\PY{l+s+s1}{\PYZsq{}}\PY{p}{,} \PY{n}{lw}\PY{o}{=}\PY{l+m+mi}{5}\PY{p}{,} \PY{n}{alpha}\PY{o}{=}\PY{l+m+mf}{0.6}\PY{p}{,} \PY{n}{label}\PY{o}{=}\PY{l+s+s1}{\PYZsq{}}\PY{l+s+s1}{norm pdf}\PY{l+s+s1}{\PYZsq{}}\PY{p}{)}
         \PY{n}{plt}\PY{o}{.}\PY{n}{show}\PY{p}{(}\PY{p}{)}
\end{Verbatim}

    And compare the histogram:

    \begin{Verbatim}[commandchars=\\\{\}]
{\color{incolor}In [{\color{incolor}37}]:} \PY{n}{fig}\PY{p}{,} \PY{n}{ax} \PY{o}{=} \PY{n}{plt}\PY{o}{.}\PY{n}{subplots}\PY{p}{(}\PY{l+m+mi}{1}\PY{p}{,} \PY{l+m+mi}{1}\PY{p}{)}
         \PY{n}{ax}\PY{o}{.}\PY{n}{hist}\PY{p}{(}\PY{n}{r}\PY{p}{,} \PY{n}{normed}\PY{o}{=}\PY{k+kc}{True}\PY{p}{,} \PY{n}{histtype}\PY{o}{=}\PY{l+s+s1}{\PYZsq{}}\PY{l+s+s1}{stepfilled}\PY{l+s+s1}{\PYZsq{}}\PY{p}{,} \PY{n}{alpha}\PY{o}{=}\PY{l+m+mf}{0.2}\PY{p}{,} \PY{n}{label}\PY{o}{=}\PY{l+s+s1}{\PYZsq{}}\PY{l+s+s1}{...}\PY{l+s+s1}{\PYZsq{}}\PY{p}{)}
         \PY{n}{ax}\PY{o}{.}\PY{n}{legend}\PY{p}{(}\PY{n}{loc}\PY{o}{=}\PY{l+s+s1}{\PYZsq{}}\PY{l+s+s1}{best}\PY{l+s+s1}{\PYZsq{}}\PY{p}{,} \PY{n}{frameon}\PY{o}{=}\PY{k+kc}{False}\PY{p}{)}
         \PY{n}{plt}\PY{o}{.}\PY{n}{show}\PY{p}{(}\PY{p}{)}
\end{Verbatim}

    \section{Linear regression model
(线性回归模型)}\label{linear-regression-model-ux7ebfux6027ux56deux5f52ux6a21ux578b}

This example computes a least-squares regression for two sets of
measurements.

    \begin{Verbatim}[commandchars=\\\{\}]
{\color{incolor}In [{\color{incolor}38}]:} \PY{k+kn}{from} \PY{n+nn}{scipy} \PY{k}{import} \PY{n}{stats}
         \PY{k+kn}{import} \PY{n+nn}{numpy} \PY{k}{as} \PY{n+nn}{np}
         \PY{n}{x} \PY{o}{=} \PY{n}{np}\PY{o}{.}\PY{n}{random}\PY{o}{.}\PY{n}{random}\PY{p}{(}\PY{l+m+mi}{10}\PY{p}{)}
         \PY{n}{y} \PY{o}{=} \PY{n}{np}\PY{o}{.}\PY{n}{random}\PY{o}{.}\PY{n}{random}\PY{p}{(}\PY{l+m+mi}{10}\PY{p}{)}
         \PY{n}{slope}\PY{p}{,} \PY{n}{intercept}\PY{p}{,} \PY{n}{r\PYZus{}value}\PY{p}{,} \PY{n}{p\PYZus{}value}\PY{p}{,} \PY{n}{std\PYZus{}err} \PY{o}{=} \PY{n}{stats}\PY{o}{.}\PY{n}{linregress}\PY{p}{(}\PY{n}{x}\PY{p}{,}\PY{n}{y}\PY{p}{)}
         \PY{n+nb}{print}\PY{p}{(}\PY{p}{\PYZob{}}\PY{l+s+s1}{\PYZsq{}}\PY{l+s+s1}{slope}\PY{l+s+s1}{\PYZsq{}}\PY{p}{:}\PY{n}{slope}\PY{p}{,}\PY{l+s+s1}{\PYZsq{}}\PY{l+s+s1}{intercept}\PY{l+s+s1}{\PYZsq{}}\PY{p}{:}\PY{n}{intercept}\PY{p}{\PYZcb{}}\PY{p}{)}
         \PY{n+nb}{print}\PY{p}{(}\PY{p}{\PYZob{}}\PY{l+s+s1}{\PYZsq{}}\PY{l+s+s1}{p\PYZus{}value}\PY{l+s+s1}{\PYZsq{}}\PY{p}{:}\PY{n}{p\PYZus{}value}\PY{p}{,}\PY{l+s+s1}{\PYZsq{}}\PY{l+s+s1}{r\PYZhy{}squared}\PY{l+s+s1}{\PYZsq{}}\PY{p}{:}\PY{n+nb}{round}\PY{p}{(}\PY{n}{r\PYZus{}value}\PY{o}{*}\PY{o}{*}\PY{l+m+mi}{2}\PY{p}{,}\PY{l+m+mi}{2}\PY{p}{)}\PY{p}{\PYZcb{}}\PY{p}{)}
\end{Verbatim}

    \begin{Verbatim}[commandchars=\\\{\}]
\{'slope': -0.16344304227778697, 'intercept': 0.60919656607207551\}
\{'p\_value': 0.65616905736353337, 'r-squared': 0.029999999999999999\}
    \end{Verbatim}

    \subsection{Optimization (优化)}\label{optimization-ux4f18ux5316}

The \texttt{minimize} function provides a common interface to
unconstrained and constrained minimization algorithms for multivariate
scalar functions in \texttt{scipy.optimize}

    \begin{Verbatim}[commandchars=\\\{\}]
{\color{incolor}In [{\color{incolor}39}]:} \PY{k+kn}{import} \PY{n+nn}{numpy} \PY{k}{as} \PY{n+nn}{np}
         \PY{k+kn}{from} \PY{n+nn}{scipy}\PY{n+nn}{.}\PY{n+nn}{optimize} \PY{k}{import} \PY{n}{minimize}
         
         \PY{c+c1}{\PYZsh{}\PYZsh{} Define the function}
         \PY{k}{def} \PY{n+nf}{rosen}\PY{p}{(}\PY{n}{x}\PY{p}{)}\PY{p}{:}
             \PY{l+s+sd}{\PYZdq{}\PYZdq{}\PYZdq{}The Rosenbrock function\PYZdq{}\PYZdq{}\PYZdq{}}
             \PY{k}{return} \PY{n+nb}{sum}\PY{p}{(}\PY{l+m+mf}{100.0}\PY{o}{*}\PY{p}{(}\PY{n}{x}\PY{p}{[}\PY{l+m+mi}{1}\PY{p}{:}\PY{p}{]}\PY{o}{\PYZhy{}}\PY{n}{x}\PY{p}{[}\PY{p}{:}\PY{o}{\PYZhy{}}\PY{l+m+mi}{1}\PY{p}{]}\PY{o}{*}\PY{o}{*}\PY{l+m+mf}{2.0}\PY{p}{)}\PY{o}{*}\PY{o}{*}\PY{l+m+mf}{2.0} \PY{o}{+} \PY{p}{(}\PY{l+m+mi}{1}\PY{o}{\PYZhy{}}\PY{n}{x}\PY{p}{[}\PY{p}{:}\PY{o}{\PYZhy{}}\PY{l+m+mi}{1}\PY{p}{]}\PY{p}{)}\PY{o}{*}\PY{o}{*}\PY{l+m+mf}{2.0}\PY{p}{)}
         
         \PY{n}{x0} \PY{o}{=} \PY{n}{np}\PY{o}{.}\PY{n}{array}\PY{p}{(}\PY{p}{[}\PY{l+m+mf}{1.3}\PY{p}{,} \PY{l+m+mf}{0.7}\PY{p}{,} \PY{l+m+mf}{0.8}\PY{p}{,} \PY{l+m+mf}{1.9}\PY{p}{,} \PY{l+m+mf}{1.2}\PY{p}{]}\PY{p}{)}
         
         \PY{c+c1}{\PYZsh{}\PYZsh{} Calling the minimize() function}
         \PY{n}{res} \PY{o}{=} \PY{n}{minimize}\PY{p}{(}\PY{n}{rosen}\PY{p}{,} \PY{n}{x0}\PY{p}{,} \PY{n}{method}\PY{o}{=}\PY{l+s+s1}{\PYZsq{}}\PY{l+s+s1}{nelder\PYZhy{}mead}\PY{l+s+s1}{\PYZsq{}}\PY{p}{,}
                        \PY{n}{options}\PY{o}{=}\PY{p}{\PYZob{}}\PY{l+s+s1}{\PYZsq{}}\PY{l+s+s1}{xtol}\PY{l+s+s1}{\PYZsq{}}\PY{p}{:} \PY{l+m+mi}{1}\PY{n}{e}\PY{o}{\PYZhy{}}\PY{l+m+mi}{8}\PY{p}{,} \PY{l+s+s1}{\PYZsq{}}\PY{l+s+s1}{disp}\PY{l+s+s1}{\PYZsq{}}\PY{p}{:} \PY{k+kc}{True}\PY{p}{\PYZcb{}}\PY{p}{)}
         \PY{n+nb}{print}\PY{p}{(}\PY{n}{res}\PY{o}{.}\PY{n}{x}\PY{p}{)}
\end{Verbatim}

    \begin{Verbatim}[commandchars=\\\{\}]
Optimization terminated successfully.
         Current function value: 0.000000
         Iterations: 339
         Function evaluations: 571
[ 1.  1.  1.  1.  1.]
    \end{Verbatim}

    \section{Data Visualizing
(数据可视化)}\label{data-visualizing-ux6570ux636eux53efux89c6ux5316}

    \begin{Verbatim}[commandchars=\\\{\}]
{\color{incolor}In [{\color{incolor}43}]:} \PY{k+kn}{from} \PY{n+nn}{matplotlib} \PY{k}{import} \PY{n}{pyplot} \PY{k}{as} \PY{n}{plt}
         \PY{n}{years} \PY{o}{=} \PY{p}{[}\PY{l+m+mi}{1950}\PY{p}{,} \PY{l+m+mi}{1960}\PY{p}{,} \PY{l+m+mi}{1970}\PY{p}{,} \PY{l+m+mi}{1980}\PY{p}{,} \PY{l+m+mi}{1990}\PY{p}{,} \PY{l+m+mi}{2000}\PY{p}{,} \PY{l+m+mi}{2010}\PY{p}{]}
         \PY{n}{gdp} \PY{o}{=} \PY{p}{[}\PY{l+m+mf}{300.2}\PY{p}{,} \PY{l+m+mf}{543.3}\PY{p}{,} \PY{l+m+mf}{1075.9}\PY{p}{,} \PY{l+m+mf}{2862.5}\PY{p}{,} \PY{l+m+mf}{5979.6}\PY{p}{,} \PY{l+m+mf}{10289.7}\PY{p}{,} \PY{l+m+mf}{14958.3}\PY{p}{]}
         \PY{c+c1}{\PYZsh{} create a line chart, years on x\PYZhy{}axis, gdp on y\PYZhy{}axis}
         \PY{n}{fig} \PY{o}{=} \PY{n}{plt}\PY{o}{.}\PY{n}{figure}\PY{p}{(}\PY{p}{)}
         \PY{n}{plt}\PY{o}{.}\PY{n}{plot}\PY{p}{(}\PY{n}{years}\PY{p}{,} \PY{n}{gdp}\PY{p}{,} \PY{n}{color}\PY{o}{=}\PY{l+s+s1}{\PYZsq{}}\PY{l+s+s1}{green}\PY{l+s+s1}{\PYZsq{}}\PY{p}{,} \PY{n}{marker}\PY{o}{=}\PY{l+s+s1}{\PYZsq{}}\PY{l+s+s1}{o}\PY{l+s+s1}{\PYZsq{}}\PY{p}{,} \PY{n}{linestyle}\PY{o}{=}\PY{l+s+s1}{\PYZsq{}}\PY{l+s+s1}{solid}\PY{l+s+s1}{\PYZsq{}}\PY{p}{)}
         \PY{c+c1}{\PYZsh{} add a title}
         \PY{n}{plt}\PY{o}{.}\PY{n}{title}\PY{p}{(}\PY{l+s+s2}{\PYZdq{}}\PY{l+s+s2}{Nominal GDP}\PY{l+s+s2}{\PYZdq{}}\PY{p}{)}
         \PY{c+c1}{\PYZsh{} add a label to the y\PYZhy{}axis}
         \PY{n}{plt}\PY{o}{.}\PY{n}{ylabel}\PY{p}{(}\PY{l+s+s2}{\PYZdq{}}\PY{l+s+s2}{Billions of \PYZdl{}}\PY{l+s+s2}{\PYZdq{}}\PY{p}{)}
         \PY{n}{plt}\PY{o}{.}\PY{n}{show}\PY{p}{(}\PY{p}{)}
\end{Verbatim}

    \subsection{3D Plot (3D绘图)}\label{d-plot-3dux7ed8ux56fe}

    \begin{Verbatim}[commandchars=\\\{\}]
{\color{incolor}In [{\color{incolor}41}]:} \PY{k+kn}{from} \PY{n+nn}{scipy} \PY{k}{import} \PY{n}{special}
         \PY{k}{def} \PY{n+nf}{drumhead\PYZus{}height}\PY{p}{(}\PY{n}{n}\PY{p}{,} \PY{n}{k}\PY{p}{,} \PY{n}{distance}\PY{p}{,} \PY{n}{angle}\PY{p}{,} \PY{n}{t}\PY{p}{)}\PY{p}{:}
            \PY{n}{kth\PYZus{}zero} \PY{o}{=} \PY{n}{special}\PY{o}{.}\PY{n}{jn\PYZus{}zeros}\PY{p}{(}\PY{n}{n}\PY{p}{,} \PY{n}{k}\PY{p}{)}\PY{p}{[}\PY{o}{\PYZhy{}}\PY{l+m+mi}{1}\PY{p}{]}
            \PY{k}{return} \PY{n}{np}\PY{o}{.}\PY{n}{cos}\PY{p}{(}\PY{n}{t}\PY{p}{)} \PY{o}{*} \PY{n}{np}\PY{o}{.}\PY{n}{cos}\PY{p}{(}\PY{n}{n}\PY{o}{*}\PY{n}{angle}\PY{p}{)} \PY{o}{*} \PY{n}{special}\PY{o}{.}\PY{n}{jn}\PY{p}{(}\PY{n}{n}\PY{p}{,} \PY{n}{distance}\PY{o}{*}\PY{n}{kth\PYZus{}zero}\PY{p}{)}
         \PY{n}{theta} \PY{o}{=} \PY{n}{np}\PY{o}{.}\PY{n}{r\PYZus{}}\PY{p}{[}\PY{l+m+mi}{0}\PY{p}{:}\PY{l+m+mi}{2}\PY{o}{*}\PY{n}{np}\PY{o}{.}\PY{n}{pi}\PY{p}{:}\PY{l+m+mi}{50}\PY{n}{j}\PY{p}{]}
         \PY{n}{radius} \PY{o}{=} \PY{n}{np}\PY{o}{.}\PY{n}{r\PYZus{}}\PY{p}{[}\PY{l+m+mi}{0}\PY{p}{:}\PY{l+m+mi}{1}\PY{p}{:}\PY{l+m+mi}{50}\PY{n}{j}\PY{p}{]}
         \PY{n}{x} \PY{o}{=} \PY{n}{np}\PY{o}{.}\PY{n}{array}\PY{p}{(}\PY{p}{[}\PY{n}{r} \PY{o}{*} \PY{n}{np}\PY{o}{.}\PY{n}{cos}\PY{p}{(}\PY{n}{theta}\PY{p}{)} \PY{k}{for} \PY{n}{r} \PY{o+ow}{in} \PY{n}{radius}\PY{p}{]}\PY{p}{)}
         \PY{n}{y} \PY{o}{=} \PY{n}{np}\PY{o}{.}\PY{n}{array}\PY{p}{(}\PY{p}{[}\PY{n}{r} \PY{o}{*} \PY{n}{np}\PY{o}{.}\PY{n}{sin}\PY{p}{(}\PY{n}{theta}\PY{p}{)} \PY{k}{for} \PY{n}{r} \PY{o+ow}{in} \PY{n}{radius}\PY{p}{]}\PY{p}{)}
         \PY{n}{z} \PY{o}{=} \PY{n}{np}\PY{o}{.}\PY{n}{array}\PY{p}{(}\PY{p}{[}\PY{n}{drumhead\PYZus{}height}\PY{p}{(}\PY{l+m+mi}{1}\PY{p}{,} \PY{l+m+mi}{1}\PY{p}{,} \PY{n}{r}\PY{p}{,} \PY{n}{theta}\PY{p}{,} \PY{l+m+mf}{0.5}\PY{p}{)} \PY{k}{for} \PY{n}{r} \PY{o+ow}{in} \PY{n}{radius}\PY{p}{]}\PY{p}{)}
\end{Verbatim}

    \begin{Verbatim}[commandchars=\\\{\}]
{\color{incolor}In [{\color{incolor}42}]:} \PY{k+kn}{import} \PY{n+nn}{matplotlib}\PY{n+nn}{.}\PY{n+nn}{pyplot} \PY{k}{as} \PY{n+nn}{plt}
         \PY{k+kn}{from} \PY{n+nn}{mpl\PYZus{}toolkits}\PY{n+nn}{.}\PY{n+nn}{mplot3d} \PY{k}{import} \PY{n}{Axes3D}
         \PY{k+kn}{from} \PY{n+nn}{matplotlib} \PY{k}{import} \PY{n}{cm}
         \PY{n}{fig} \PY{o}{=} \PY{n}{plt}\PY{o}{.}\PY{n}{figure}\PY{p}{(}\PY{p}{)}
         \PY{n}{ax} \PY{o}{=} \PY{n}{Axes3D}\PY{p}{(}\PY{n}{fig}\PY{p}{)}
         \PY{n}{ax}\PY{o}{.}\PY{n}{plot\PYZus{}surface}\PY{p}{(}\PY{n}{x}\PY{p}{,} \PY{n}{y}\PY{p}{,} \PY{n}{z}\PY{p}{,} \PY{n}{rstride}\PY{o}{=}\PY{l+m+mi}{1}\PY{p}{,} \PY{n}{cstride}\PY{o}{=}\PY{l+m+mi}{1}\PY{p}{,} \PY{n}{cmap}\PY{o}{=}\PY{n}{cm}\PY{o}{.}\PY{n}{jet}\PY{p}{)}
         \PY{n}{ax}\PY{o}{.}\PY{n}{set\PYZus{}xlabel}\PY{p}{(}\PY{l+s+s1}{\PYZsq{}}\PY{l+s+s1}{X}\PY{l+s+s1}{\PYZsq{}}\PY{p}{)}
         \PY{n}{ax}\PY{o}{.}\PY{n}{set\PYZus{}ylabel}\PY{p}{(}\PY{l+s+s1}{\PYZsq{}}\PY{l+s+s1}{Y}\PY{l+s+s1}{\PYZsq{}}\PY{p}{)}
         \PY{n}{ax}\PY{o}{.}\PY{n}{set\PYZus{}zlabel}\PY{p}{(}\PY{l+s+s1}{\PYZsq{}}\PY{l+s+s1}{Z}\PY{l+s+s1}{\PYZsq{}}\PY{p}{)}
         \PY{n}{plt}\PY{o}{.}\PY{n}{show}\PY{p}{(}\PY{p}{)}
\end{Verbatim}


    % Add a bibliography block to the postdoc
    
    
    
    \end{document}
